\documentclass{beamer} 

\usepackage[utf8]{inputenc}
\usepackage[portuguese]{babel}
\usetheme{Madrid}
\usecolortheme{default}
\usepackage{ragged2e}

\usepackage{array}
\usepackage{amsmath}
\usepackage{amssymb}
\usepackage{amsthm}
\usepackage{mathtools}

\title{Grupos Quocientes}
\subtitle{Subgrupos Normais e Estruturas Quocientes}
\author{Marco Busetti}
\institute{UTFPR}
\date{\today}

\AtBeginSection[]{
  \frame{
    \frametitle{Tabela de Conteúdo}
    \tableofcontents[currentsection]
  }
}

\begin{document}

\frame{\titlepage}

\begin{frame}
\frametitle{Roteiro}
\tableofcontents
\end{frame}

\section{Conjunto Quociente e Operações}

\begin{frame}
\frametitle{Conjunto Quociente}

\begin{block}{Definição}
Seja $G$ um grupo e $H \leq G$.

Chamamos de \textbf{conjunto quociente} o conjunto $G/H$ cujos elementos são as classes laterais à esquerda de $H$ em $G$.
\end{block}

\pause

\begin{block}{Observação}
Decorre da definição que $|G/H| = [G:H]$ (índice de $H$ em $G$).
\end{block}

\end{frame}

\begin{frame}
\frametitle{Operação entre Classes Laterais}

\begin{block}{Definição}
Seja $G$ um grupo e $H \leq G$. Definimos a seguinte operação:
\[
\bullet : G/H \times G/H \to G/H
\]
\[
(xH, yH) \mapsto xyH
\]
\end{block}

\pause

\begin{alertblock}{Problema}
Esta operação nem sempre está bem definida!

Para que $xH \bullet yH = xyH$ esteja bem definida, precisamos que representantes diferentes da mesma classe lateral produzam o mesmo resultado.
\end{alertblock}

\end{frame}

\section{Subgrupos Normais}

\begin{frame}
\frametitle{Condições para Operação Bem Definida}

\begin{alertblock}{Proposição}
Seja $G$ um grupo e $H \leq G$. As seguintes afirmações são equivalentes:

\begin{enumerate}
\item A operação $\bullet$ está bem definida
\item $gHg^{-1} \subseteq H$, $\forall g \in G$
\item $gHg^{-1} = H$, $\forall g \in G$
\item $gH = Hg$, $\forall g \in G$
\end{enumerate}
\end{alertblock}

\pause

\textbf{Ideia da demonstração:} A operação está bem definida se e somente se elementos conjugados por qualquer $g \in G$ permanecem no subgrupo $H$.

\end{frame}

\begin{frame}
\frametitle{Subgrupos Normais}

\begin{block}{Definição}
Um subgrupo $H$ é um \textbf{subgrupo normal} de $G$ caso satisfaça as condições equivalentes da proposição anterior.

Denotamos: $H \trianglelefteq G$
\end{block}

\pause

\begin{exampleblock}{Observações}
\begin{itemize}
\item Se $H \trianglelefteq G$, então classes laterais à esquerda e à direita são iguais
\item $H \triangleleft G$ denota subgrupo normal próprio
\item Para verificar normalidade: mostrar que $ghg^{-1} \in H$
\end{itemize}
\end{exampleblock}

\end{frame}

\begin{frame}
\frametitle{Exemplos de Subgrupos Normais}

\begin{exampleblock}{Exemplos}
\begin{itemize}
\item $G$ e $\{e\}$ são sempre subgrupos normais (triviais)
\pause
\item O centro $Z(G) \trianglelefteq G$ para qualquer grupo $G$
\pause
\item $SL_n(\mathbb{K}) \trianglelefteq GL_n(\mathbb{K})$
\pause
\item Se $[G:H] = 2$, então $H \trianglelefteq G$
\pause
\item Em grupos abelianos, todos os subgrupos são normais
\end{itemize}
\end{exampleblock}

\end{frame}

\begin{frame}
\frametitle{Grupos Simples}

\begin{block}{Definição}
Um grupo $G$ não-trivial é chamado de \textbf{grupo simples} se seus únicos subgrupos normais são $\{e\}$ e $G$.
\end{block}

\pause

\begin{exampleblock}{Interpretação}
Grupos simples são os "blocos de construção" da teoria de grupos - não podem ser "decompostos" em grupos menores via quocientes não-triviais.
\end{exampleblock}

\end{frame}

\section{Operações com Subgrupos}

\begin{frame}
\frametitle{Produto de Subgrupos}

\begin{block}{Definição}
Sejam $A, B \leq G$. Definimos:
\[
AB = \{ab : a \in A, b \in B\}
\]
\end{block}

\pause

\begin{alertblock}{Proposição}
$HK$ é subgrupo de $G$ se e somente se $HK = KH$.
\end{alertblock}

\pause

\begin{exampleblock}{Consequência}
Se $H \trianglelefteq G$ ou $K \trianglelefteq G$, então $HK \leq G$.
\end{exampleblock}

\end{frame}

\begin{frame}
\frametitle{Cardinalidade do Produto}

\begin{alertblock}{Proposição (Grupos Finitos)}
Seja $G$ um grupo finito e $H, K \leq G$. Então:
\[
|HK| = \frac{|H||K|}{|H \cap K|}
\]
\end{alertblock}

\pause

\begin{exampleblock}{Corolário}
Se $HK \leq G$, então:
\[
[HK:K] = [H:H \cap K]
\]
\end{exampleblock}

\end{frame}

\section{Grupos Quocientes}

\begin{frame}
\frametitle{Definição de Grupo Quociente}

\begin{alertblock}{Teorema}
Seja $G$ um grupo e $H \trianglelefteq G$.

Então $(G/H, \bullet)$ é um grupo, chamado de \textbf{grupo quociente}.
\end{alertblock}

\pause

\textbf{Verificação dos axiomas:}
\begin{itemize}
\item \textbf{Associatividade:} $(xH \bullet yH) \bullet zH = xH \bullet (yH \bullet zH)$
\item \textbf{Elemento neutro:} $H$
\item \textbf{Elemento inverso:} $(xH)^{-1} = x^{-1}H$
\end{itemize}

\end{frame}

\begin{frame}
\frametitle{Exemplo Principal}

\begin{exampleblock}{$\mathbb{Z}/n\mathbb{Z}$}
O grupo $\mathbb{Z}/n\mathbb{Z}$ é um grupo quociente formado por:
\begin{itemize}
\item $G = \mathbb{Z}$ (grupo aditivo dos inteiros)
\item $H = n\mathbb{Z}$ (múltiplos de $n$)
\item Classes: $\{0, 1, 2, \ldots, n-1\} + n\mathbb{Z}$
\end{itemize}
\end{exampleblock}

\pause

\textbf{Operação:} $\overline{a} + \overline{b} = \overline{a+b}$

onde $\overline{a} = a + n\mathbb{Z}$

\end{frame}

\begin{frame}
\frametitle{Propriedade do Centro}

\begin{alertblock}{Proposição}
Seja $G$ um grupo e $Z(G)$ seu centro.

Se $G/Z(G)$ é cíclico, então $G = Z(G)$ (isto é, $G$ é abeliano).
\end{alertblock}

\pause

\textbf{Ideia da demonstração:}
\begin{itemize}
\item Se $G/Z(G) = \langle xZ(G) \rangle$, então todo elemento de $G$ é da forma $x^k z$
\item Elementos desta forma comutam entre si
\item Logo $G$ é abeliano, portanto $G = Z(G)$
\end{itemize}

\end{frame}

\begin{frame}
\frametitle{Obrigado!}

\begin{center}
\Large
Dúvidas?
\end{center}

\end{frame}

\begin{frame}
\frametitle{Referências}

\begin{itemize}
\item GARCIA, Arnaldo; LEQUAIN, Yves. \textit{Elementos de Álgebra}. Rio de Janeiro: IMPA, 2003.
\item https://github.com/MARCOVB5/grupos
\end{itemize}

\end{frame}

\end{document}
