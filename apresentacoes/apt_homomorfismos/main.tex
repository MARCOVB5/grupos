\documentclass{beamer} 

\usepackage[utf8]{inputenc}
\usepackage[portuguese]{babel}
\usetheme{Madrid}
\usecolortheme{default}

\usepackage{array}
\usepackage{amsmath}
\usepackage{amssymb}
\usepackage{amsthm}
\usepackage{mathtools}
\usepackage{pict2e}

\title{Homomorfismos de Grupos}
\subtitle{Definições, Propriedades e Teoremas de Isomorfismo}
\author{Marco Busetti}
\institute{UTFPR}
\date{\today}

\AtBeginSection[]{
  \frame{
    \frametitle{Tabela de Conteúdo}
    \tableofcontents[currentsection]
  }
}

\newcommand{\trianglechar}{
  \mathrel{
    \unitlength=1mm
    \begin{picture}(5,3)(0,-1.5)
      \polyline(0,0)(4,1.5)(4,-1.5)(0,0)
      \polyline(1.25,0)(3,0.75)(3,-0.75)(1.25,0)
    \end{picture}
  }
}

\newcommand{\trianglechareq}{
  \mathrel{
    \unitlength=1mm
    \begin{picture}(5,3)(0,-1.5)
      \polyline(0,0)(4,1.5)(4,-1.5)(0,0)
      \polyline(1.25,0)(3,0.75)(3,-0.75)(1.25,0)
      \polyline(0,-1.9)(4,-1.9)
    \end{picture}
  }
}

\begin{document}

\frame{\titlepage}

\begin{frame}
\frametitle{Roteiro}
\tableofcontents
\end{frame}

\section{Definição e Exemplos}

\begin{frame}
\frametitle{Definição de Homomorfismo}

\begin{block}{Definição}
Sejam $(G, \cdot)$ e $(\mathcal{G}, *)$ dois grupos. Uma função $\varphi : G \rightarrow \mathcal{G}$ é chamada de \textbf{homomorfismo de grupos} se:
\[
\varphi(a\cdot b) = \varphi(a)*\varphi(b), \; \forall a,b \in G
\]
\end{block}

\pause

\begin{exampleblock}{Propriedades Imediatas}
Se $\varphi$ é um homomorfismo, então:
\begin{enumerate}
	\item $\varphi(e_G) = e_\mathcal{G}$
	\item $\varphi(x^{-1}) = \varphi(x)^{-1}, \; \forall x \in G$
\end{enumerate}
\end{exampleblock}

\end{frame}

\begin{frame}
\frametitle{Exemplos de Homomorfismos}

\begin{exampleblock}{Exemplos Comuns}
\begin{itemize}
    \item \textbf{Homomorfismo Trivial:} $\theta: G \to \mathcal{G}$ dado por $\theta(g) = e_{\mathcal{G}}, \; \forall g\in G$.
    
    \pause
    \item \textbf{Determinante:} A função $\det: GL_n(\mathbb{K}) \to \mathbb{K}^*$ é um homomorfismo, pois $\det(AB) = \det(A)\det(B)$.
    
    \pause
    \item \textbf{Projeção Canônica:} Se $N \trianglelefteq G$, a função $\pi: G \to G/N$ dada por $\pi(x) = xN$ é um homomorfismo sobrejetivo.
\end{itemize}
\end{exampleblock}

\end{frame}

\section{Núcleo e Imagem}

\begin{frame}
\frametitle{Núcleo de um Homomorfismo}

\begin{block}{Definição (Núcleo)}
Definimos o \textbf{núcleo} (ou kernel) de um homomorfismo $\varphi: G \to \mathcal{G}$ como o subconjunto:
\[
\ker \varphi \overset{\text{def}}{=} \{x \in G \mid \varphi(x) = e_\mathcal{G}\}
\]
\end{block}

\pause

\begin{alertblock}{Proposição Fundamental}
O núcleo de um homomorfismo $\varphi$ é sempre um \textbf{subgrupo normal} de $G$.
\[
\ker \varphi \trianglelefteq G
\]
\end{alertblock}

\pause

\begin{exampleblock}{Exemplo}
Para o homomorfismo $\det: GL_n(\mathbb{K}) \to \mathbb{K}^*$:
\[
\ker(\det) = \{A \in GL_n(\mathbb{K}) \mid \det(A) = 1\} = SL_n(\mathbb{K})
\]
\end{exampleblock}

\end{frame}

\begin{frame}
\frametitle{Imagem de um Homomorfismo}

\begin{block}{Definição (Imagem)}
Definimos a \textbf{imagem} de um homomorfismo $\varphi: G \to \mathcal{G}$ como:
\[
\operatorname{Im} \varphi \overset{\text{def}}{=} \{ y \in \mathcal{G} \mid \exists x \in G,\, y = \varphi(x) \}
\]
\end{block}

\pause

\begin{alertblock}{Proposição}
A imagem de um homomorfismo $\varphi$ é sempre um \textbf{subgrupo} de $\mathcal{G}$.
\[
\operatorname{Im} \varphi \leq \mathcal{G}
\]
(Note: A imagem não é, em geral, um subgrupo normal de $\mathcal{G}$.)
\end{alertblock}

\end{frame}

\section{Isomorfismos}

\begin{frame}
\frametitle{Monomorfismos (Homomorfismos Injetivos)}

\begin{block}{Definição}
Um homomorfismo $\varphi$ que é \textbf{injetivo} é chamado de \textbf{monomorfismo}.
\end{block}

\pause

\begin{alertblock}{Teste de Injetividade}
Um homomorfismo $\varphi$ é injetivo se, e somente se, seu núcleo contém apenas o elemento neutro:
\[
\varphi \text{ é injetiva} \iff \ker \varphi = \{e_G\}
\]
\end{alertblock}

\end{frame}

\begin{frame}
\frametitle{Isomorfismos (Grupos "Iguais")}

\begin{block}{Definição}
Um homomorfismo $\varphi : G \to \mathcal{G}$ é um \textbf{isomorfismo} se $\varphi$ for \textbf{bijetivo} (injetivo e sobrejetivo).
\end{block}

\pause

\begin{block}{Grupos Isomorfos}
Se existe um isomorfismo entre $G$ e $\mathcal{G}$, dizemos que os grupos são \textbf{isomorfos} e denotamos $G \cong \mathcal{G}$.
\end{block}

\pause

\begin{exampleblock}{Exemplo Clássico}
$(\mathbb{R}, +) \cong (\mathbb{R}^{*}_{+}, \cdot)$
\begin{itemize}
    \item O isomorfismo pode ser $\varphi(x) = \log(x)$ ou $\psi(x) = e^x$.
    \item $\varphi(x \cdot y) = \log(xy) = \log(x) + \log(y) = \varphi(x) + \varphi(y)$
\end{itemize}
\end{exampleblock}

\end{frame}

\section{Teoremas de Isomorfismo}

\begin{frame}
\frametitle{Primeiro Teorema dos Isomorfismos}

\begin{alertblock}{Teorema}
Seja $\varphi : G \to \mathcal{G}$ um homomorfismo de grupos. Então, temos um isomorfismo:
\[
\frac{G}{\ker \varphi} \cong \operatorname{Im} \varphi
\]
\end{alertblock}

\pause

\begin{block}{Interpretação}
\begin{itemize}
    \item Este é talvez o teorema mais importante da teoria de grupos.
    \item Ele conecta os dois conceitos centrais que vimos:
    \begin{enumerate}
        \item \textbf{Grupos Quocientes} (via $\ker \varphi$, que é normal)
        \item \textbf{Subgrupos} (via $\operatorname{Im} \varphi$)
    \end{enumerate}
\end{itemize}
\end{block}

\end{frame}

\begin{frame}
\frametitle{Segundo e Terceiro Teoremas}

\begin{block}{Segundo Teorema dos Isomorfismos}
Seja $G$ um grupo, $H \trianglelefteq G$ e $K \leq G$. Então:
\[
\frac{K}{H\cap K} \cong \frac{HK}{H}
\]
\end{block}

\pause

\begin{block}{Terceiro Teorema dos Isomorfismos}
Seja $G$ um grupo e $H, K \trianglelefteq G$ tal que $K \subseteq H$. Então $H/K \trianglelefteq G/K$ e:
\[
\frac{G/K}{H/K} \cong \frac{G}{H}
\]
(Este teorema é sobre "cancelar" o subgrupo $K$).
\end{block}

\end{frame}

\section{Automorfismos}

\begin{frame}
\frametitle{Automorfismos}

\begin{block}{Definição}
Um \textbf{automorfismo} de $G$ é um isomorfismo $\varphi: G \to G$.
\end{block}

\pause

\begin{block}{Grupo de Automorfismos}
O conjunto de todos os automorfismos de $G$, denotado por $\operatorname{Aut}(G)$, forma um grupo sob a operação de composição de funções $(\circ)$.
\end{block}

\pause

\begin{exampleblock}{Exemplo: Automorfismo Interno}
Para qualquer $g \in G$, a função "conjugação por $g$":
\[
\mathcal{I}_g(x) = gxg^{-1}
\]
é um automorfismo de $G$.
\end{exampleblock}

\end{frame}

\begin{frame}
\frametitle{Automorfismos Internos e Subgrupos}

\begin{block}{Grupo de Automorfismos Internos}
O conjunto de todos os automorfismos internos, $\operatorname{Inn}(G)$, é um subgrupo de $\operatorname{Aut}(G)$.
\end{block}

\pause

\begin{alertblock}{Proposição}
O grupo de automorfismos internos é um subgrupo normal do grupo de automorfismos:
\[
\operatorname{Inn}(G) \trianglelefteq \operatorname{Aut}(G)
\]
\end{alertblock}

\pause

\begin{block}{Subgrupos Característicos}
Um subgrupo $H \leq G$ é \textbf{característico} ($H \trianglechareq G$) se ele é invariante por \textit{todos} os automorfismos (não apenas os internos).
\begin{itemize}
    \item Exemplo: O centro $Z(G)$ é sempre característico.
    \item $H \trianglechareq G \implies H \trianglelefteq G$.
\end{itemize}
\end{block}

\end{frame}

\begin{frame}
\frametitle{Obrigado!}

\begin{center}
\Large
Dúvidas?
\end{center}

\end{frame}

\begin{frame}
\frametitle{Referências}

\begin{itemize}
\item GARCIA, Arnaldo; LEQUAIN, Yves. \textit{Elementos de Álgebra}. Rio de Janeiro: IMPA, 2003.
\item Material de IC: \textit{IC Grupos: Iniciação Científica em Teoria de Grupos}
\item \url{https://github.com/MARCOVB5/grupos}
\end{itemize}

\end{frame}

\end{document}
