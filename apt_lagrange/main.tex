\documentclass{beamer} 

\usepackage[utf8]{inputenc}
\usepackage[portuguese]{babel}
\usetheme{Madrid}
\usecolortheme{default}
\usepackage{ragged2e}

\usepackage{array}
\usepackage{amsmath}
\usepackage{amssymb}
\usepackage{amsthm}
\usepackage{mathtools}

\title{Teorema de Lagrange}
\subtitle{No Contexto de Teoria de Grupos}
\author{Marco Busetti}
\institute{UTFPR}
\date{\today}

\AtBeginSection[]{
  \frame{
    \frametitle{Tabela de Conteúdo}
    \tableofcontents[currentsection]
  }
}

\begin{document}

\frame{\titlepage}

\begin{frame}
\frametitle{Roteiro}
\tableofcontents
\end{frame}

\section{Um Pequeno Histórico}

\begin{frame}
\frametitle{A pré-história do Teorema de Lagrange}
\begin{justify}
\fontsize{15}{16}\selectfont
Lagrange, em seus escritos de 1770/71, introduziu uma ideia importantíssima que, ao longo de um século, se desenvolveu no que hoje conhecemos como Teorema de Lagrange. Embora o teorema moderno conhecido em Teoria de Grupos — que afirma que a ordem de um subgrupo de um grupo finito sempre divide a ordem do grupo — não estivesse presente em seu trabalho original, sua percepção inicial foi o alicerce para essa descoberta fundamental.
\end{justify}
\end{frame}

\begin{frame}
\fontsize{15}{16}\selectfont
\frametitle{Os $m$-valores para funções de $n$ variáveis}
\begin{justify}
Supondo $f$ uma função bem definida de $n$ entradas, definimos $m$ como o número de diferentes funções obtidas em permutar as $n$ variáveis de entrada. Isto é, os 'valores' de $f$.
\end{justify}
Denotamos:
\begin{itemize} 
\item $n\;\rightarrow$ um inteiro positivo;

\item $x_1, x_2, x_3, \ldots, x_n\;\rightarrow$ $n$ variáveis;

\item $f(x_1,x_2,x_3, \ldots, x_n)\;\rightarrow$ uma função avaliada em $n$ variáveis;

\item $m\; \rightarrow$ o número de diferentes funções obtidas via permutação das variáveis $x_1, x_2, x_3, \ldots, x_n$.

\end{itemize}
{\color{blue} Lagrange:} Dado $n$, quais são os diferentes valores que $m$ pode ter? 

\end{frame}

\begin{frame}
\frametitle{Valores possíveis para $m$ e a conjectura de Lagrange}
\begin{center}
\begin{tabular}{|c|c|}
\hline
\textbf{n} & \textbf{m (valores possíveis para $f$)} \\
\hline
1 &  1 \\
\hline 
2 &  1, 2 \\
\hline
3 &  1, 2, 3, 6 \\
\hline
4 & 1, 2, 3, 4, 6, 8, 12, 24 \\
\hline
5 &  {\color{red} E quanto a esta linha ??}\\
\hline
\end{tabular}
\end{center}

\begin{itemize}
\item Na seção 97 de \textit{Réflexions} de Lagrange, é conjecturado:
\end{itemize}

\begin{center}
\boxed{
\text{O número $m$ de valores possíveis para $f$ divide $n!$}
}
\end{center}
\end{frame}

\begin{frame}
\frametitle{Teorema de Lagrange: 1770 e 1870}
\fontsize{15}{16}\selectfont
\begin{justify}

{\color{blue} Teorema de Lagrange ~1770:} o número $m$ de valores possíveis de uma função de $n$ varíaveis divide $n!$

\vspace{0.7cm}


{\color{blue} Teorema de Lagrange ~1870:} Si le groupe $H$ est contenu dans le
groupe $G$, son ordre $n$ est un diviseur de $N$, ordre de $G$. 

[{\color{green}Camille Jordan,} \textit{Traité des Substitutions et des Équations Algébriques} (1870), p. 25— onde foi dada a origem do nome ao teorema que conhecemos hoje como Teorema de Lagrange.

\vspace{0.7cm}

\textbf{{\color{blue} A pergunta central:} Como esses teoremas se conectam??}

\end{justify}

\end{frame}

\section{Conceitos Fundamentais}

\begin{frame}
\frametitle{Operação Binária}

\begin{block}{Definição}
Sejam $G$ e $E$ conjuntos não-vazios e $\oplus$ uma função tal que:
\[
\oplus: 
\begin{array}{c}
G\times G \to E \\
(a, b) \mapsto a \oplus b
\end{array}
\]
Definimos $\oplus$ como uma \textbf{operação binária} de dois elementos de $G$ em $E$.
\end{block}

\pause

\begin{block}{Definição}
Dizemos que $\oplus$ é uma \textbf{lei de composição interna} em $G$ se $E = G$.
\end{block}

\end{frame}

\begin{frame}
\frametitle{Definição de Grupo}

\begin{block}{Definição}
Seja $G$ um conjunto não-vazio. Dizemos que $(G, \cdot)$ é um \textbf{grupo} se, e somente se, $\cdot$ é uma lei de composição interna em $G$ tal que:

\begin{enumerate}
\item \textbf{Elemento neutro:} $\exists \,e \in G \; , \forall \,x \in G \; : x\cdot e = e\cdot x = x$
\item \textbf{Elemento inverso:} $\forall x \in G, \exists x^{-1}\in G: \; x\cdot x^{-1} = x^{-1} \cdot x = e$
\item \textbf{Associatividade:} $\forall \, x,y,z \in G : (x\cdot y)\cdot z = x\cdot (y\cdot z)$
\end{enumerate}
\end{block}

\pause

\textbf{Observação:} Se $\forall \,(x,y) \in G \times G:\; x \cdot y = y \cdot x$, dizemos que $G$ é um grupo \emph{abeliano} (ou \emph{comutativo}).

\end{frame}

\begin{frame}
\frametitle{Exemplos de Grupos}

\begin{itemize}
\item $(\mathbb{Z}, +)$, $(\mathbb{R}, +)$, $(\mathbb{C}, +)$ - grupos abelianos
\pause
\item $(\mathbb{R}^*, \cdot)$, $(\mathbb{C}^*, \cdot)$, $(\mathbb{Q}^*, \cdot)$ - grupos abelianos
\pause
\item $(GL_n(\mathbb{K}), \times)$ - grupo das matrizes $n\times n$ invertíveis
\pause
\item $(\mathbb{Z}/n\mathbb{Z}, +)$ - grupos cíclicos finitos
\pause
\item $(\mathbb{Z}_p^*, \odot)$ - grupo multiplicativo módulo $p$ primo
\pause
\item $S_n$ - grupo simétrico (permutações de $n$ elementos)
\end{itemize}

\end{frame}

\section{Subgrupos}

\begin{frame}
\frametitle{Definição de Subgrupo}

\begin{block}{Definição}
Seja $(G, \cdot)$ um grupo. Um subconjunto $H \subseteq G$ é chamado de \textbf{subgrupo} de $G$ (denotamos $H \leq G$) se, e somente se, $(H, \cdot)$ é um grupo.
\end{block}

\pause

\begin{alertblock}{Teorema}
Seja $H \subseteq G$ tal que $H \neq \emptyset$ e $(G, \cdot)$ é um grupo. Então $H \leq G$ se, e somente se:
\begin{enumerate}
\item $h_1 \cdot h_2 \in H,\; \forall (h_1,h_2) \in H \times H$
\item $h^{-1} \in H, \; \forall h\in H$
\end{enumerate}
\end{alertblock}

\end{frame}

\begin{frame}
\frametitle{Exemplos de Subgrupos}

\begin{itemize}
\item $G$ e $\{e\}$ são subgrupos \textbf{triviais} de $G$
\pause
\item $(n\mathbb{Z}, +) \leq (\mathbb{Z}, +)$ para todo $n \in \mathbb{Z}$
\pause
\item $SL_n(\mathbb{K}) = \{A \in GL_n(\mathbb{K}) : \det(A) = 1\} \leq GL_n(\mathbb{K})$
\pause
\item Centro do grupo: $Z(G) = \{x \in G : xg = gx, \; \forall g \in G\}$
\end{itemize}

\end{frame}

\begin{frame}
\frametitle{Subgrupo Gerado}

\begin{block}{Definição}
Seja $(G, \cdot)$ um grupo e $X \subseteq G$ não-vazio. O \textbf{subgrupo gerado por $X$} é:
\[
\langle X \rangle = \bigcap \{H : H \leq G \text{ e } X \subseteq H\}
\]
\end{block}

\pause

\begin{exampleblock}{Proposição}
\[
\langle X \rangle = \{x_1 x_2 \ldots x_n : x_i \in X \cup X^{-1}, \; n \geq 1\}
\]
\end{exampleblock}

\pause

Para um único elemento: $\langle a \rangle = \{a^n : n \in \mathbb{Z}\}$

\end{frame}

\begin{frame}
\frametitle{Grupos Cíclicos}

\begin{block}{Definição}
Um grupo $G$ é chamado de \textbf{cíclico} quando pode ser gerado por um único elemento $a \in G$, isto é, $G = \langle a \rangle$.
\end{block}

\pause

\begin{exampleblock}{Proposição}
Se $G$ é um grupo cíclico, então $G$ é abeliano.
\end{exampleblock}

\pause

\textbf{Exemplos:}
\begin{itemize}
\item $\mathbb{Z} = \langle 1 \rangle$
\item $\mathbb{Z}/n\mathbb{Z} = \langle \overline{1} \rangle$
\end{itemize}

\end{frame}

\begin{frame}
\frametitle{Ordem de Elementos}

\begin{block}{Definição}
Seja $(G, \cdot)$ um grupo.
\begin{itemize}
\item A \textbf{ordem do grupo} $G$ é $|G|$ (número de elementos)
\item A \textbf{ordem de um elemento} $\alpha \in G$ é $\mathcal{O}(\alpha) = |\langle \alpha \rangle|$
\end{itemize}
\end{block}

\pause

\begin{exampleblock}{Proposição}
Seja $G$ um grupo e $\alpha \in G$. São equivalentes:
\begin{enumerate}
\item $\mathcal{O}(\alpha) < \infty$
\item $\exists t \in \mathbb{Z}^*_+ : \alpha^t = e$ (onde $t$ é minimal)
\end{enumerate}
\end{exampleblock}

\end{frame}

\section{Classes Laterais}

\begin{frame}
\frametitle{Classes Laterais}

\begin{block}{Definição}
Seja $G$ um grupo, $H \leq G$ e $x \in G$.
\begin{itemize}
\item \textbf{Classe lateral à esquerda:} $xH = \{xh : h \in H\}$
\item \textbf{Classe lateral à direita:} $Hx = \{hx : h \in H\}$
\end{itemize}
\end{block}

\pause

\begin{block}{Definição}
O \textbf{índice de $H$ em $G$} é o número de classes laterais distintas:
\[
[G:H] = |\{\text{classes laterais à esquerda de } H\}|
\]
\end{block}

\pause

\textbf{Propriedade importante:} O número de classes laterais à esquerda é igual ao número de classes laterais à direita.

\end{frame}

\begin{frame}
\frametitle{Propriedades das Classes Laterais}

\begin{exampleblock}{Proposição}
Seja $G$ um grupo, $H \leq G$ e $x, y \in G$. Então:
\begin{enumerate}
\item $x \in xH$ (todo elemento está em sua classe lateral)
\pause
\item $xH = yH$ se, e somente se, $x^{-1}y \in H$
\pause
\item Ou $xH = yH$ ou $xH \cap yH = \emptyset$
\pause
\item $|xH| = |H|$ para todo $x \in G$
\end{enumerate}
\end{exampleblock}

\pause

\textbf{Conclusão:} As classes laterais formam uma \textbf{partição} de $G$.

\end{frame}

\section{Teorema de Lagrange}

\begin{frame}
\frametitle{Teorema de Lagrange}

\begin{alertblock}{Teorema de Lagrange}
Seja $G$ um grupo finito e $H$ um subgrupo de $G$.

Então $|H|$ divide $|G|$.

Mais precisamente: $|G| = |H| \cdot [G:H]$
\end{alertblock}

\pause

\textbf{Ideia da Demonstração:}
\begin{itemize}
\item As classes laterais de $H$ particionam $G$
\item Cada classe lateral tem cardinalidade $|H|$
\item Existem $[G:H]$ classes laterais distintas
\item Logo: $|G| = |H| \cdot [G:H]$
\end{itemize}

\end{frame}

\begin{frame}
\frametitle{Consequências do Teorema de Lagrange}

\begin{exampleblock}{Corolário}
Se $G$ é um grupo finito e $\alpha \in G$, então $\mathcal{O}(\alpha)$ divide $|G|$.
\end{exampleblock}

\pause

\begin{exampleblock}{Corolário}
Se $G$ é um grupo finito de ordem $p$ primo, então $G$ é cíclico.
\end{exampleblock}

\pause

\begin{alertblock}{Teorema de Euler}
Seja $G$ um grupo finito com $|G| = n$. Então:
\[
\forall g \in G, \quad g^n = e
\]
\end{alertblock}

\end{frame}

\begin{frame}
\frametitle{Pequeno Teorema de Fermat}

\begin{alertblock}{Pequeno Teorema de Fermat}
Seja $p$ um número primo e $a \in \mathbb{Z} \setminus p\mathbb{Z}$. Então:
\[
a^{p-1} \equiv 1 \pmod{p}
\]
\end{alertblock}

\pause

\textbf{Demonstração:} Aplicação direta do Teorema de Euler ao grupo $(\mathbb{Z}_p^*, \odot)$, que tem ordem $p-1$.

\end{frame}

\begin{frame}
\frametitle{Multiplicatividade do Índice}

\begin{exampleblock}{Proposição}
Seja $G$ um grupo e sejam $K \leq H \leq G$. Então:
\[
[G:K] = [G:H] \cdot [H:K]
\]
\end{exampleblock}

\pause

\textbf{Demonstração:} Aplicação sucessiva do Teorema de Lagrange:
\begin{align}
|G| &= |H| \cdot [G:H] \\
|H| &= |K| \cdot [H:K] \\
|G| &= |K| \cdot [G:K]
\end{align}

\end{frame}

\section{Voltando ao Teorema de 1770}

\begin{frame}

{\color{blue} Teorema de Lagrange (1770):} O número $m$ de valores de uma função $f$ de $n$ variáveis divide n!
\vspace{0.7cm}

Seja $G = S_n$ o grupo de todas as permutações dos elementos em $\{x_1, x_2, \ldots, x_n\}$.

Seja $f^g$ a função obtida ao aplicar a permutação $g\in G$ nas variáveis em $f$.

Tomemos $H$ como a família de todas as permutações $h\in G$ tais que $f^h = f$.

Logo, $H$ é subgrupo de $G$ e 2 'valores' de $f$, $f^{g_1}$ e $f^{g_2}$ são iguais se e somente se $g_1$ e $g_2$ estão na mesma classe lateral de $H$.

Portanto, o número de 'valores' de $f$ é o número de classes laterais de $H$ em $G$ e, logo, divide $n!$

\qed 

\end{frame}

\section{Agradecimentos/Referências}

\begin{frame}
\frametitle{Obrigado!}

\begin{center}
\Large
Dúvidas?
\end{center}

\end{frame}

\begin{frame}
\frametitle{Referências}

\begin{itemize}
\item https://m-a.org.uk/resources/downloads/3H-Peter-Neumann-Lagrange-Theorem.pdf
\item https://github.com/MARCOVB5/grupos
\item GARCIA, Arnaldo; LEQUAIN, Yves. \textit{Elementos de Álgebra}. Rio de Janeiro: IMPA, 2003. 
\end{itemize}

\end{frame}

\end{document}
