\documentclass[16pt,openany]{book}

\usepackage[a4paper, margin=3cm]{geometry}
\usepackage{pagenote}
\usepackage{anyfontsize}
\usepackage[portuguese]{babel}
\usepackage{amsmath}    % Ambientes e símbolos matemáticos básicos
\usepackage{amssymb}    % Símbolos matemáticos adicionais
\usepackage{amsfonts}   % Fontes matemáticas adicionais
\usepackage{amsthm}
\usepackage{mathtools}  % Extensão do amsmath com funcionalidades extras
\usepackage{bm}         % Vetores e matrizes em negrito
\usepackage{mathrsfs}   % Fontes script caligráficas
\usepackage{esint}      % Integrais múltiplas e notações extras
\usepackage{cancel}     % Símbolos de cancelamento
\usepackage{nicefrac}   % Frações em linha no formato texto
\usepackage{siunitx}    % Escrita de unidades e notação científica
\usepackage{array}      % Formatação avançada de tabelas e matrizes
\usepackage{arydshln}   % Linhas tracejadas em tabelas e matrizes
\usepackage{ytableau}   % Tabelas de Young e outras notações combinatórias
\usepackage{tcolorbox}
\tcbuselibrary{breakable}
\usepackage{xcolor}
\usepackage{hyperref}
\definecolor{lightred}{rgb}{1,0.6,0.6}

% fontes AMS EULER !!! 
% \usepackage{eulervm} 
%\usepackage[T1]{fontenc} 
%\usepackage{mathpazo}

\theoremstyle{definition}

\newtheorem{definition}{Definição}[section]  % Definições numeradas por subseção
\newtheorem{example}{Exemplo}[section]       % Exemplos numerados por subseção
\newtheorem{theorem}{Teorema}[section]     % Teoremas numerados como as definições
\newtheorem{proposition}{Proposição}[section] % Proposições numeradas como as definições
\newtheorem{corollary}{Colorário}[section] % Colorários numerados como as definições

\newtcolorbox{mybox}[2]{%
  colback=#1!10!white, % Fundo colorido
  colframe=#1!50!black, % Borda mais escura
  boxrule=0.5pt,        % Espessura da borda
  sharp corners,        % Bordas retas
  fonttitle=\bfseries,  % Título em negrito
  breakable,            % Permite quebra de página
  title={\textbf{#2}},  % Define o título
}

% Redefinindo os ambientes para adicionar caixinhas com numeração
\renewenvironment{definition}[1][]{%
  \refstepcounter{definition}%
  \begin{mybox}{green}{Definição \thesection.\arabic{definition}}%
}{\end{mybox}}

\renewenvironment{example}[1][]{%
  \refstepcounter{example}%
  \begin{mybox}{yellow}{Exemplo \thesection.\arabic{example}}%
}{\end{mybox}}

\renewenvironment{theorem}[1][]{%
  \refstepcounter{theorem}%
  \begin{mybox}{blue}{Teorema \thesection.\arabic{theorem}}%
}{\end{mybox}}

\renewenvironment{proposition}[1][]{%
  \refstepcounter{proposition}%
  \begin{mybox}{red}{Proposição \thesection.\arabic{proposition}}%
}{\end{mybox}}

\renewenvironment{corollary}[1][]{%
  \refstepcounter{corollary}%
  \begin{mybox}{lightred}{Corolário \thesection.\arabic{corollary}}%
}{\end{mybox}}


\newcommand{\titlepagecontent}{
    \begin{center}
        \vspace*{2cm} 
        {\Huge\bfseries IC Grupos} \\[1cm]
        {\Large Iniciação Científica em Teoria de Grupos} \\[2cm]
        
        {\Large Marco Vieira Busetti} \\[0.5cm]
        {\Large Professor: Francismar Ferreira Lima} \\[2cm]
        
        {\Large Universidade Tecnológica Federal do Paraná} \\[0.5cm]
        {\Large Curitiba, Novembro de 2024}
    \end{center}
}

\newtheoremstyle{no-italic}  % Nome do estilo
  {3pt}   % Espaçamento antes
  {3pt}   % Espaçamento depois
  {\normalfont} % Fonte do corpo (normal, sem itálico)
  {}      % Identação
  {\bfseries} % Fonte do título
  {.}     % Ponto após o título
  { }     % Espaçamento após o título
  {}


\begin{document}

\begin{titlepage}
    \titlepagecontent
\end{titlepage}

\fontsize{12}{14}\selectfont

\chapter{Generalidades sobre Grupos}

\section{Operações Binárias}

\begin{definition}

Sejam $G$ e $E$ conjuntos não-vazios e $\oplus$ uma função tal que:

\[
\oplus: 
\begin{array}{c}
G\bigtimes G \to E \\
(a, b) \mapsto \oplus (a,b)
\end{array}
\]

Definimos a função acima como a \textbf{operação binária de dois elementos de $G$ em $E$} e a escrevemos comumente como: $a\oplus b$.
\end{definition}

\begin{example}
    A adição usual $+$ é uma operação binária de dois elementos de $\mathbb{I}$ em $\mathbb{R}$. Onde $\mathbb{I}$ denota o conjunto dos números irracionais.
\end{example}

\begin{example}
    Sejam $(a,b)\in \mathbb{R}^2$, a função que define a distância cartesiana entre dois pontos $a$ e $b$:
    \[
    \text{dist}(a,b): 
    \begin{array}{c}
    \mathbb{R}\bigtimes \mathbb{R} \to \mathbb{R}^+ \\
    (a, b) \mapsto \sqrt{a^2 + b^2}
    \end{array}
    \]
    representa uma operação binária de dois elementos de $\mathbb{R}$ em $\mathbb{R}^+$.
    
\end{example}

\begin{definition}
A partir das notações acima, definimos \textbf{lei de composição interna de $G\times G \to G$ se $E = G$}.
\end{definition}

\textit{Observação: caso não haja ambiguidade, denotaremos simplesmente \textbf{lei de composição interna em $G$} para representar a lei de composição interna de $G\times G \to G$.}

\begin{example}
    A operação usual $+$ em $\mathbb{N}$ é uma lei de composição interna em $\mathbb{N}$, ao contrário da operação usual $-$ de $\mathbb{N}$ em $\mathbb{Z}$.
\end{example}

\section{Grupos}

\begin{definition}
    Seja $G$ um conjunto não-vazio. \textbf{Dizemos que $(G, \cdot)$ é um grupo} se, e somente se, $\cdot$ é uma lei de composição interna em $G$ tal que:
    \begin{enumerate}
    \item $\exists \,e \in G \; , \forall \,x \in G \; : x\cdot e = e\cdot x = x $;
    \item $\forall x \in G, \exists \hat{x}\in G: \; x\cdot \hat{x} = \hat{x} \cdot x = e$;
    \item $\forall \, x,y,z \in G : (x\cdot y)\cdot z = x\cdot (y\cdot z)$.
    \end{enumerate}
\end{definition}

\textit{Observações:}

\textit{Levando em consideração as notações acima, temos:}
\begin{enumerate}
	\item \textit{Primeiramente, notamos que $e$ e $\hat{x}$ são únicos, uma vez que:}

    \textit{Supondo que existam $e$ e $e'$ pertencentes à $G$ que satisfazem o item 1, temos:}
    \[
        x \cdot e = x = x \cdot e' 
        \;\;\;\implies\;\;\;
        \hat{x}\cdot x \cdot e 
        = \hat{x} \cdot x \cdot e' 
        \;\;\;\implies\;\;\;
        e = e' 
        \quad \qedsymbol
    \]

    \textit{Supondo agora que existam $\hat{x}$ e $\hat{x}'$ que satisfaçam o item 2, temos:}
    \[
        \hat{x}\cdot x 
        = e 
        = \hat{x}'\cdot x 
        \implies
        \hat{x}\cdot x \cdot \hat{x} 
        = \hat{x}'\cdot x \cdot \hat{x} 
        \implies
        \hat{x}\cdot e 
        = \hat{x}'\cdot e 
        \implies
        \hat{x} = \hat{x}' 
        \quad \qedsymbol
    \]
    \item \textit{Notamos por convenção $x^{-1}$ no lugar de $\hat{x}$ no \textbf{item 2} (dada sua unicidade).}
    
    \item \textit{Caso $\forall \,(x,y) \in G \times G:\; x \cdot y = y \cdot x$, dizemos que $G$ é um grupo \emph{abeliano} (ou \emph{comutativo}).}
    
    \item \textit{Caso $G$ seja um grupo abeliano, então} 
    \[
      (x\cdot y)^n = x^n \cdot y^n, 
      \quad \forall n \in \mathbb{Z}.
    \]
\end{enumerate}

\begin{example}
$(\mathbb{Z, +}), \;(\mathbb{Z}/n\mathbb{Z}, +),\;(\mathbb{R}^*, \cdot), \; (\mathbb{R}, +), \; (\mathbb{C, +}),\; (\mathbb{C^*, \cdot}), \; (\mathbb{Q}^*, \cdot)$ \\ são grupos abelianos (onde $+$ e $\cdot$ denotam as operações usuais de adição e produto em $\mathbb{C}$).
\end{example}

\begin{example}
    $(GL_n(\mathbb{K}), \times)$ define uma estrutura de grupo, onde $\mathbb{K}= \mathbb{C}$ ou $\mathbb{R}$ e $GL_n(\mathbb{K})$ define o conjunto das matrizes $n\times n$ invertíveis com entradas em $\mathbb{K}$.
\end{example}

\begin{example}
    Seja $A$ um conjunto não-vazio. Seja 
    \[
    \mathcal{P}(f)  = \{f:A \to A\; \vert\; f\;\text{bijetiva} \}
    \]
    O conjunto das funções $f$ bijetivas de $A$ em $A$.\\
    $(\mathcal{P}(f), \circ )$ define uma estrutura de grupo, onde $\circ$ representa composição entre funções. \\
    Caso $A$ seja um conjunto finito e $n\in \mathbb{N}$ tal que $\text{Card}(A) = n$, $\mathcal{P}(f)$ será representado por $S_n$ e será chamado de \textbf{grupo simétrico ou grupo das permutações}.
\end{example}

\begin{example}
Seja, neste exemplo, para fins de simplificação, $\mathbb{Z} / n\mathbb{Z} \overset{def}{=} \mathbb{Z}_n$, para $n\in \mathbb{Z}$.

Seja a operação $\odot$ em $\mathbb{Z}_n$ definida da seguinte forma:

\[
\odot: 
\begin{array}{c}
\mathbb{Z}_n \bigtimes \mathbb{Z}_n \to \mathbb{Z}_n \\
(\overline{a}, \overline{b}) \mapsto \overline{a} \odot \overline{b} = \overline{a\cdot b}
\end{array}
\] 

onde $\cdot$ é a operação usual de produto nos inteiros.

Temos que $(\mathbb{Z}_{p}^{*}, \odot)$, onde $p$ é um número primo, é um grupo abeliano.

\end{example}
\textbf{Demonstração:}

Por construção, temos que $\overline{a}\odot \overline{b} \in \mathbb{Z}_{p}^{*}$.

Para mostrar a associatividade, sejam $\overline{a}, \overline{b}, \overline{c} \in \mathbb{Z}_{p}^{*}$.

Temos que:

\begin{align*}
	&\overline{a}\odot (\overline{b}\odot \overline{c}) = \overline{a}\odot (\overline{b\cdot c}) =\\[10pt]
	&= \overline{a\cdot (b\cdot c)} = \overline{(a\cdot b)\cdot c} = (\overline{a}\odot \overline{b})\odot \overline{c}.
\end{align*}

O elemento neutro é evidentemente o elemento $\overline{1}\in \mathbb{Z}_{p}^{*}$, pois:

\[
\overline{a}\odot \overline{1} = \overline{a\cdot 1} = \overline{a}, \; \forall \overline{a} \in \mathbb{Z}_{p}^{*}.
\]

Também temos que para todo elemento de $\mathbb{Z}_{p}^{*}$, existe elemento inverso, pois, sabemos que:

\[
\forall \overline{a}\in \mathbb{Z}_{p}^{*} \implies \text{mdc}(a, p) = 1.
\]

Logo, pelo Teorema de Bézout, temos que existem $x$ e $y$ inteiros tais que:

\[
ax - py = 1
\]

Ora mas isso é a mesma coisa que afirmar que existe uma solução para a equação: 

\[
a \cdot x \equiv 1 \pmod{p} \iff \overline{a} \odot \overline{x} = \overline{1}. 
\]

Logo, deduzimos que $\forall \overline{a} \in \mathbb{Z}_{p}^{*}, \; \exists \overline{a}^{-1} \in \mathbb{Z}_{p}^{*}$.

Além disso, é evidente que a operação $\odot$ é comutativa.

Portanto, provamos que $(\mathbb{Z}_{p}^{*}, \odot)$ é um grupo abeliano.

\qed


\begin{example}
     Seja $G = \;]-1, 1[, \; (G, \star)$ tal que 
     \[
     \forall x,y \in G \; : x\star y = \frac{x + y}{1 + xy}
     \]
     define um grupo abeliano.
\end{example}

\textbf{Demonstração:}

Provemos primeiramente que $\forall x, y \in G, x\star y \in G$.

Fixando $y \in G$ temos a seguinte função de $x \in G$:

\[
f(x) = \frac{x + y}{1 + xy}
\]

A função é derivável em $G$. Tomando sua derivada temos:

\[
f'(x) = \frac{1 - y^2}{(1+xy)^2} 
\]

Temos evidentemente $\forall (x, y) \in G \times G, f'(x) > 0$.

(De forma simétrica podemos mostrar o mesmo escrevendo $f$ como uma função de $y$).

Logo, deduzimos que a função $f$ é estritamente crescente.

Portanto:

\[
f(-1)<x \star y < f(1) \iff \frac{y-1}{1-y} < x \star y < \frac{1+y}{1+y} \iff -1 <x\star y < 1
\]

Logo, provamos que $x\star y \in G$.

Provemos os outros axiomas:

\textit{Existência do neutro}:

Tomando $y = 0$ temos:

\[
x\star 0 = \frac{x + 0}{1 + 0\cdot x} = x
\]

Portanto, deduzimos que o elemento neutro do grupo $G$ é dado por $e = 0$.

\textit{Existência do inverso}:

Tomando $y = -x$ temos:

\[
x\star -x = \frac{x -x}{1 - (-x)x} = 0
\]

Portanto, deduzimos que o elemento inverso do grupo $G$ existe e é dado por $x^{-1} = -x$.

\textit{Associatividade}:

Sejam $x, y, z \in G$, mostremos que $(x\star y)\star z = x\star (y\star z)$

Temos:

\begin{align*}
(x\star y)\star z &= \frac{(x\star y) + z}{1+(x\star y)z} = \frac{\frac{x+y}{1+xy} + z}{1 + z\frac{x+y}{1 +xy}} = \frac{x + y+z+xyz}{1+xy+xz+yz} = \\[10pt]
&= \frac{x(1+yz) + (y +z)}{(1+yz)+x(y+z)} = \frac{x+\frac{y+z}{1+yz}}{1+x\frac{y+z}{1+yz}} = x\star (y\star z)
\end{align*}

Mostrando, assim, a associatividade.

Ainda, temos que o grupo é evidentemente abeliano.\qed

\section{Subgrupos}

\begin{definition}
    Seja $(G, \cdot)$ um grupo. \textbf{Um subconjunto $H \subseteq G$ é chamado de subgrupo de $G$} (denotamos $H\leq G$) se, e somente se, $(H, \cdot)$ é um grupo.
\end{definition}
\textit{Observação: temos ainda que se $H \subset G$, temos então $H$ é chamado de subgrupo próprio de $G$ e denotamos como $H< G$.}

\begin{proposition} \hypertarget{prop131}
    Seja $H \subseteq G$ tal que $H \ne \emptyset$ e $(G, \cdot)$ é um grupo. $H\leq G$ é equivalente à satisfazer as seguintes condições:
    \begin{enumerate}
    \item $h_1 \cdot h_2 \in H,\; \forall (h_1,h_2) \in H \times H$;
    \item $h^{-1} \in H, \; \forall h\in H$.
    \end{enumerate}
\end{proposition}

\textbf{Demonstração:}

É necessário mostrarmos as duas implicações da equivalência:

\begin{align}
    H\leq G \Longrightarrow (1.)\; \text{e} \;(2.)
\end{align}

\begin{align}
    (1.)\; \text{e} \;(2.) \Longrightarrow H\leq G
\end{align}

A implicação $(1.1)$ é trivial. Ora, se $H\leq G$, então pela definição de subgrupo temos que $h_1\cdot h_2 \in H$ e $h^{-1}\in H$, isto é $\exists h^{-1} \in H : h\cdot h^{-1} = h^{-1} \cdot h = h$.

Para a implicação $(1.2)$:

Sabemos que $H \subseteq G$, logo, se $h_1 \cdot h_2 \in H \Longrightarrow h_1 \cdot h_2 \in G$. Ora, sabemos que $(G, \cdot)$ é um grupo. Logo, a associatividade é satisfeita. Para demonstrar que $e \in H$, basta tomarmos $h_2 = h^{-1}$ a partir de $(2.)$. Logo, temos $h\cdot h^{-1} = e \in H$. Com isso mostramos todos os axiomas necessários e deduzimos que $H\leq G$. \qed 

\begin{example}
    $(\mathbb{U}^*, \cdot), \;(\mathbb{R}^*, \cdot),\;(\mathbb{R}^*_+, \cdot), \; (\mathbb{Q}^*, \cdot), \; (\mathbb{Q}^*_+, \cdot)$ são subgrupos de $(\mathbb{C}^*, \cdot)$, onde $\cdot$ denota a multplicação usual em $\mathbb{C}$.
\end{example}

\begin{example}
    $G$ e $\{e\}$ são subgrupos \textit{triviais} de $G$.
\end{example}

\begin{example}
    Seja $n\in \mathbb{Z}$, $(n\mathbb{Z}, +)$ são subgrupos de $(\mathbb{Z},+)$, e, em particular, são os únicos.
\end{example}
\textbf{Demonstração:}

É evidente que $(n\mathbb{Z}, +)$ são subgrupos de $(\mathbb{Z}, +)$. Mostremos que são os únicos! 

Seja $(H, +)$ um subgrupo qualquer de $(\mathbb{Z}, +)$. Se $H = \{0\}$, então $H = 0\mathbb{Z}$.

Suponhamos agora $H \neq \{0\}$. Seja $n = \text{min}\{a \in H,\; a > 0 \}$. 

Logo, como $n \in H$ e $H \leq \mathbb{Z}$, temos que $n\mathbb{Z} \subseteq H$. 

De maneira inversa, seja $h \in H$. Logo, pelo Algoritmo de Euclides, existem $q, r \in \mathbb{Z}$ tais que:

\[
h = qn + r \; \; (0 \leq r < n)
\] 

Porém, note que, como $h \in H$, temos:

\[
r = h - qn \in H  
\]

Porém, sabemos que $0 \leq r < n$.

Ora, como $n$ é o elemento mínimo de $H$ estritamente maior que $0$, deduzimos que apenas podemos ter $r = 0$.

Logo:

\[
h = qn \implies h \in n\mathbb{Z} \implies H \subseteq n\mathbb{Z}.
\]

Portanto deduzimos que $H = n\mathbb{Z}$. \qed

\begin{example}
    Seja $G$ um grupo e $I$ um conjunto não-vazio de índices. Se $\{H_i\}_{i \in I}$ é uma família de subgrupos de $G$, então $\bigcap_{i\in I} H _i$ é um subgrupo de $G$.
\end{example}

\textbf{Demonstração:}

Como visto na \hyperlink{prop131}{\textbf{Proposição 1.3.1}}, mostremos que:

\begin{enumerate}
    \item $\forall\; x_1, x_2 \in \bigcap_{i\in I} H _i \implies x_1\cdot x_2 \in \bigcap_{i\in I} H _i$;
    \item $\forall x \in \bigcap_{i \in I} H_i \implies \exists \; x^{-1} \in \bigcap_{i\in I} H _i$.
\end{enumerate}

Provemos o \textbf{item 1}:

Sejam,

\[
x_1, x_2 \in \bigcap_{i\in I} H_i
\]

Logo:

\[
\forall i \in I, \; x_1, x_2 \in H_i
\]

Sabemos também que:

\[
\forall i \in I, \; H_i \leq G 
\]

Portanto, deduzimos que:

\[
\forall i \in I, \; x_1 \cdot x_2 \in H_i
\]

Mas isso é a mesma coisa que dizer:

\[
\forall x_1, x_2 \in \bigcap_{i \in I} H_i \implies x_1 \cdot x_2 \in \bigcap_{i\in I} H_i
\]

Provemos o \textbf{item 2}:

Analogamente ao \textbf{item 1}, sabemos que:

\[
x_0 \in \bigcap_{i \in I} H_i \iff \forall i \in I, \; x_0 \in H_i
\]

Porém, sabemos que:

\[
\forall i \in I, \; H_i \leq G
\]

Logo, deduzimos que:

\[
\forall i \in I, \; x_0 \in H_i, \; \exists x^{-1}_{0} \in H_i   
\]

Mas isso é a mesma coisa que:

\[
\forall x \in \bigcap_{i \in I} H_i \implies \exists  x^{-1} \in \bigcap_{i \in I} H_i
\]

Portanto, provamos que:

\[
\bigcap_{i \in I} H_i \leq G
\]  

\qed 

\begin{definition}
    Seja $G$ um grupo. O subconjunto $Z(G)$ tal que: 
    \[
    Z(G) = \{x \in G : xg = gx, \; \forall g \in G\}
    \]
    define um subgrupo de $G$ chamado \textit{centro} de $G$.
\end{definition}

\textbf{Demonstração:}

Como visto na \hyperlink{prop131}{\textbf{Proposição 1.3.1}}, para mostrar que $Z(G) \leq G$ é necessário mostrar que $x\cdot x^{-1} \in Z(G), \; \forall x \in Z(G)$.

Nota: $Z(G)$ é claramente não vazio uma vez que o elemento neutro comuta com todos elementos de $G$ e, portanto, está em $Z(G)$.

Temos que:

Se:
\[
x \in Z(G) \Rightarrow x\cdot g = g\cdot x, \; \forall g\in G.
\]

Logo, teremos:

\[
xgx^{-1} = g \Longrightarrow x^{-1}xgx^{-1} = x^{-1}g \Longrightarrow gx^{-1} =x^{-1}g, \; \forall g \in G 
\]

Portanto: 

\[
x^{-1} \in Z(G)
\]

Temos também que:

\[
x_1 \in Z(G) \Longrightarrow x_1g = gx_1, \; \forall g \in G \;\; \text{(I)}
\]

\[
x_2 \in Z(G) \Longrightarrow x_2g = gx_2, \; \forall g \in G \; \; \text{(II)}
\]

Deduzimos de (I):

\[
x_1g = gx_1 \Longrightarrow g = x^{-1}_1gx_1
\]

Substituindo em (II):

\[
x_2x^{-1}_1gx_1 = x^{-1}_1gx_1x_2 \Longrightarrow x_2x^{-1}_1x_1g = x^{-1}_1gx_1x_2 \Longrightarrow 
\]

\[
\Longrightarrow x_2g = x^{-1}_1gx_1x_2 \Longrightarrow (x_1x_2)g = g(x_1x_2)
\]

Logo, deduzimos que:

\[
(x_1, x_2)\in Z(G)\times Z(G) \Longrightarrow x_1\cdot x_2 \in Z(G)
\]

Portanto, $Z(G) \leq G$. \qed

\textit{Observação: O subgrupo centro serve o propósito de "medir a comutatividade" de um dado grupo. Por exemplo, observamos que $Z(\mathbb{Z}) = \mathbb{Z}$, $Z(GL_2 (\mathbb{R})) = \{\lambda I: \lambda \in \mathbb{R}^*\}$ e  $Z(S_n) = \{e\},\; n \geq 3$}.

\begin{definition} \hypertarget{def133}
    Seja $(G, \cdot)$ um grupo e $X$ um conjunto não-vazio tal que $X\subseteq G$. \textbf{Chamamos de subgrupo gerado por um subconjunto a interseção de todos os subgrupos de $G$ que contém $X$.}  Denotamos-o como $\langle X\rangle$.

Matematicamente temos:

\[
\langle X \rangle = \bigcap \{H : H \leq G \; \; \text{e} \; \; X \subseteq H\}
\]

\end{definition}

\begin{proposition}
A partir das notações da \hyperlink{def133}{\textbf{Definição 1.3.3}}, temos que $\langle X \rangle$ é o menor subgrupo de $G$ que contém $X$.
\end{proposition}

\textbf{Demonstração}:

Suponha que $J \leq G$ seja o menor subgrupo de $G$ tal que $X \subseteq J$.

Ora, como $J \leq G$ e $X \subseteq J$, então: $\langle X \rangle \subseteq J$.

Entretanto, também sabemos que $J$ é o menor subgrupo de $G$ tal que $X \subseteq J$.

Portanto, deduzimos que $J \subseteq H,\; \forall H : H \leq G \; \; \text{e} \; \; X \subseteq H$.

Porém, para todo $H$ subgrupo de $G$ temos que $X\subseteq H$, logo, deduzimos que $J \subseteq \langle X \rangle$.

Portanto, $J = \langle X \rangle$. 

\qed  


\begin{proposition}
A partir das notações da \hyperlink{def133}{\textbf{Definição 1.3.3}}, temos que:
\[
\langle X \rangle = \{x_1 x_2 ... x_n : x_i \in X\cup X^{-1}, \; n \geq 1 \}
\]
\end{proposition}

\textbf{Demonstração}:

Sejam:

\[
\dot{X} \overset{def}{=} \bigcap \{H : H \leq G \; \; \text{e} \; \; X \subseteq H\}
\]

\[
\bar{X} \overset{def}{=} \{x_1 x_2 ... x_n : x_i \in X\cup X^{-1}, \; n \geq 1 \}
\]

Queremos mostrar que: $\dot{X} = \bar{X}$.

Realizemos, primeiramente, algumas convenções de notação:

\[
\bar{x}_p \overset{def}{=} x_1 x_2 ... x_p, \; p\in \mathbb{Z}^{*}_{+}
\]

\[
\bar{x}_{p}^{-1} \overset{def}{=} x_{1}^{-1} x_{2}^{-1} ... x_{p}^{-1}, \; p \in \mathbb{Z}_{+}^{*}
\]

É evidente que $\bar{x}_p,\; \bar{x}_{p}^{-1} \in \bar{X}$. Assim como $\bar{x}_p \bar{x}_{p}^{-1} \in \bar{X}$, o que nos mostra que $\bar{X} \leq G$.

Mostremos que $\dot{X} \subseteq \bar{X}$:

Sabemos que: 

\[
\bar{X} = \{\bar{x}_p : x_i \in X \cup X^{-1}, p\in \mathbb{Z}_{+}^{*} \; \; \text{e} \; \; 1 \leq i \leq p \}
\]

Evidentemente temos que: 

\[
\forall x \in X \implies x \in \bar{X}
\]

Uma vez que $\bar{X} \leq G$, temos diretamente que $\dot{X} \subseteq \bar{X}$.

Isso se dá pelo fato de que $\dot{X}$ é o menor subgrupo de $G$ contendo $X$, e, como $\bar{X}$ é um subgrupo de $G$ contendo $X$, realizamos tal dedução.

Mostremos agora que $\bar{X} \subseteq \dot{X}$:

Seja $H \leq G$ tal que: 

\[
H \leq G \; \; \text{e} \; \; X \subseteq H.
\]

Ora, temos evidentemente que:

\[
\forall \bar{x}_p \in \bar{X} \implies \bar{x}_p \in H.
\]

Logo:

\[
\bar{x}_p \in H \implies \bar{x}_p \in \bigcap_{i \in I} H_i
\]

Onde $I$ é um conjunto não-vazio de índices.

Evidentemente temos então que $\bar{x}_p \in \dot{X}$.

Logo, $\bar{X} \subseteq \dot{X}$.

Portanto, mostramos que: $\bar{X} = \dot{X}$. 

\qed

\begin{example}
    Seja o grupo $(\mathbb{R}^*,\cdot)$ e o subconjunto $E\subset \mathbb{R}^*$ tal que $E = \{2\}$. O subgrupo gerado por $E$ é, portanto, $H=\{2^n, n\in \mathbb{Z}\}$.

    De forma genérica, para um grupo $G$ e um elemento $a\in G$, temos:
    $\langle a\rangle = \{a^n \vert n\in \mathbb{Z} \}$.
\end{example}

De forma geral, dado um grupo $G$, para determinarmos um subgrupo $H$ gerado por um subconjunto $X$ devemos provar os seguintes pontos: 

\begin{enumerate}
    \item $H$ é um subgrupo de $G$ 
    \item $ X\subset H$
    \item Se $H'$ é um outro subgrupo tal que $X\subset H'$, então $H \subset H'$
\end{enumerate}

\begin{definition}
    Seja $G$ um grupo. \textbf{$G$ é chamado de grupo cíclico quando ele pode ser gerado por um único elemento $x \in G$.}
\end{definition}

\begin{example}
    $\mathbb{Z} = \langle 1\rangle, \; \mathbb{Z}/n\mathbb{Z} = \langle \bar{1}\rangle, \; \mathbb{U} = \langle e^{\frac{2\pi i}{n}}\rangle$.
\end{example}

\begin{proposition} \hypertarget{prop132}
    Se $G$ é um grupo cíclico, então $G$ é um grupo abeliano.
\end{proposition}

\textbf{Demonstração:}

Seja $a \in G$ tal que $G = \langle a\rangle$. Podemos representar $G$ como:

\[
G = \{...,\; (a^{-1})^{r},\; ...,\; (a^{-1})^2, \; a^{-1},\; e, \; a,\; a^2, \; ..., a^r, \; ...\} 
\]

Onde $r \in \mathbb{Z}$.

Sejam $(x,y)\in G \times G$, queremos mostrar que $x\cdot y = y\cdot x$.

Sabemos que:

\[
x = a^{r_1}, \; r_1 \in \mathbb{Z}
\]

\[
y = a^{r_2}, \; r_2 \in \mathbb{Z}
\]

Logo:

\[
x\cdot y = a^{r_1} \cdot a^{r_2} = a^{r_1 + r_2} \stackrel{(*)}{=} a^{r_2+ r_1} = a^{r_2}\cdot a^{r_1} = y\cdot x
\]

\textit{$(*)$ : deduz-se que $r_1 + r_2 = r_2 + r_1$ pois estamos trabalhando dentro do grupo abeliano $(\mathbb{Z}, +)$}.

Portanto, $G$ é um grupo abeliano. \qed

\begin{definition}
    Definimos $\langle \{ xyx^{-1}y^{-1} \vert (x,y) \in G \times G \} \rangle$ como o subgrupo dos comutadores do grupo $G$. Denotaremos-o por $G'$.
\end{definition}

\begin{definition}
    Seja $(G, \cdot)$ um grupo. \textbf{Definimos ordem do grupo $(G, \cdot)$ a quantidade de elementos no conjunto $G$} e a denotamos por $\vert G \vert$. \\
    Se $\alpha \in G$, \textbf{a ordem de $\alpha$ é a ordem do subgrupo gerado por $\alpha$}, denotada por $\mathcal{O}(\alpha)$, isto é,  $ \mathcal{O}(\alpha) = \vert \langle \alpha \rangle \vert$.
\end{definition}

\begin{example}
    $\vert \mathbb{Z} \vert = \infty, \; \vert\mathbb{Z}/n\mathbb{Z}\vert = n, \; \vert(\mathbb{Z}/p\mathbb{Z})^* \vert = p-1, \; \vert S_n \vert = n!$
\end{example}

\begin{proposition} \hypertarget{prop135}
	Seja $G$ um grupo finito e $\alpha$ um elemento de $G$. 

	Logo, $\mathcal{O}(\alpha) < \infty$.
\end{proposition}

\textbf{Demonstração:}

Provemos a \hyperlink{prop135}{\textbf{Proposição 1.3.5}} via absurdo.

Suponha que $\mathcal{O}(\alpha)$ seja não finito, logo podemos gerar $n$ valores distintos a partir de potências de $\alpha$, onde $n \in \mathbb{Z}$.

Ora, a partir da geração de infinitos valores distintos de potências de $\alpha$, sabemos que, para dado valor inteiro $k$, teremos $\alpha ^k \notin G$. Ora, mas $\langle \alpha \rangle$ é um subgrupo de $G$. Absurdo.

Portanto, temos que $\mathcal{O}(\alpha) < \infty$. 

\qed  

\begin{proposition} \hypertarget{prop136}
Seja $G$ um grupo e $\alpha$ um elemento de $G$. Então, as seguintes proposições são equivalentes:

$(i)$ A ordem $\mathcal{O}(\alpha)$ é finita. Isto é, $\mathcal{O}(\alpha) < \infty$;

$(ii)$ $\exists t \in \mathbb{Z}_{+}^{*} : \alpha ^t = e$, onde $t = \min{\{k \in G : k > 0\}}$.

\end{proposition}

\textbf{Demonstração:}

Queremos provar que: $(i) \iff (ii)$.

Comecemos provando a implicação $(i) \Longrightarrow (ii):$

Temos, por definição, que $\langle \alpha \rangle = \{\alpha ^m \; \vert \; m \in \mathbb{Z}\}$. 

Como $\mathcal{O}(\alpha) < \infty$, temos que $\exists p, q \in \mathbb{Z} : p > q$ e $\alpha ^p = \alpha ^q$.

Deduzimos diretamente que: $\alpha ^{p-q} = e$. Como $p-q \in \mathbb{Z}^{*}_{+}$, mostramos $(i) \Longrightarrow (ii)$.

Note que a escolha do valor $p-q$ ocorre sem perda de generalidade, uma vez que o conjunto $\mathbb{Z}_{+}^{*}$ é enumerável e sempre podemos garantir a minimalidade de $p-q$. 

Provemos $(ii) \Longrightarrow (i):$

Ora, a partir de $(ii)$ sabemos que $\langle \alpha \rangle$ é finito e, pela minimalidade de $t$, sua ordem é igual à $t$.

Portanto, a partir da \hyperlink{prop135}{\textbf{Proposição 1.3.5}} temos diretamente que $\mathcal{O}(\alpha) < \infty$.

Portanto, com isso, mostramos que $(ii) \Longrightarrow (i)$ e, consequentemente, mostramos $(i) \iff (ii)$. 

\qed

\section{Teorema de Lagrange}

\begin{definition}
    Seja $G$ um grupo e $H$ um subgrupo de $G$. \textbf{Definimos classe lateral à esquerda de $H$ em $G$ que contém $x$ o subconjunto $xH$} de $G$ tal que $\forall x \in G$:
    \[
    xH = \{xh \; \vert \; h\in H \}
    \]
    Analogamente definimos \textbf{classe lateral à direita de $H$ em $G$ que contém $x$ o subconjunto $Hx$} de $G$ tal que $\forall x \in G$:
    \[
    Hx = \{ hx \; \vert \; h \in H\}
    \]
\end{definition}


\textit{Observações:}
    \begin{itemize}
        \item \textit{As classes laterais de $G$ não são necessariamente subgrupos de $G$;}
        \item \textit{Quando não houver confusão possível, podemos denominar as classes laterais à esquerda/direita de $H$ em $G$ que contém $x$ como simplesmente: classe lateral à esquerda/direita de $H$}.
    \end{itemize}

\begin{definition} \hypertarget{def142}
	A cardinalidade do conjunto das classes laterais à esquerda \textit{ou} à direita é definida como \textbf{o índice de $H$ em $G$}, e será denotada por $[G:H]$.
\end{definition} 

\textit{Observação: note que o número de classes laterais à direita de $H$ é igual ao número de classes laterais à esquerda de $H$ (por mais que as classes laterais sejam diferentes).} 

\textit{Isto se dá pelo fato de que a função:} 

\[
\phi: 
\begin{array}{c}
\{\text{classes lat. à esquerda}\} \to \{\text{classes lat. à direita} \} \\
xH \mapsto Hx^{-1}
\end{array}
\]

\textit{é claramente uma bijeção.}

\begin{theorem} \hypertarget{lagrange}
{\bfseries Teorema de Lagrange (Grupos)}\normalfont

    Seja $G$ um grupo finito e $H$ um subgrupo de $G$.

    Logo, $\vert H\vert$ divide $\vert G \vert$.
\end{theorem}

\textbf{Demonstração:}


%A ideia da demonstração se baseia em, a partir de um subgrupo $H$ de $G$, formar partições de $G$ duas-a-duas disjuntas tais que a cardinalidade dessas partições sejam iguais à ordem de $H$.

%Isto é, se $\{X_i \}_{i\in\mathbb{N}}$ representa uma família de subconjuntos de $G$ e $H$ é um subgrupo de $G$, então queremos provar que:

%\[
%\bigcup_{i \in \mathbb{N}}X_i = G; \; \; \bigcap_{i \in \mathbb{N}}X_i = \emptyset\; \; \text{e} \; \; \text{Card}(X_i) = \vert H\vert, \; \forall i \in \mathbb{N}.
%\]

Seja $x\in G \backslash H$, consideremos o conjunto das classes laterais à esquerda de $H$:

\[
xH = \{xh \; \vert \; h \in H\}
\]

Mostremos que $H\cap xH = \emptyset$:

Supondo $\alpha \in H\cap xH$:

\[
\alpha \in H\cap xH \iff \alpha = xh \in H.
\]

Como $\alpha =xh \in H$, logo $\exists h^{-1}\in H$ tal que $hh^{-1} \in H$

Portanto:

\[
\alpha h^{-1} = xhh^{-1} \in H \iff x \in H \Longrightarrow \text{Absurdo, pois   } x\in G\backslash H.
\]

Logo, $H\cap xH = \emptyset$.

Agora mostremos que $\text{Card}(xH) = \vert H\vert$:

Seja $\zeta$ a função definida abaixo:

\[
\zeta: 
\begin{array}{c}
H \to xH \\
h \mapsto xh
\end{array}
\]

A função $\zeta$ é claramente sobrejetiva por definição. 

$\zeta$ também é injetiva pois se $(xh_1, xh_2) \in (xH)^2$:

\[
xh_1 = xh_2 \Longrightarrow x^{-1}xh_1 = x^{-1}xh_2 \Longrightarrow h_1 = h_2.
\]

Portanto, deduzimos que $\text{Card}(xH) = \vert H\vert$.

Consideremos agora o conjunto $yH$ das classes laterais à esquerda de $H$ em $G$ que contém $y$ tal que $y \notin H\cup xH$.

Já mostramos anteriormente que $y \notin H$.

Mostremos que $yH \cap xH = \emptyset$

Supondo $\beta \in yH \cap xH$:

Então $\beta$ pode ser escrito de duas formas:

\[
\beta = yh_1
\]

\[
\beta = xh_2
\]

Logo, temos:

\[
yh_1 = xh_2 \Longrightarrow y = xh_2h_1^{-1} \in xH \Longrightarrow \text{Absurdo, pois  } y \notin H\cup xH.
\]

Analogamente ao passo anterior podemos provar que $\text{Card}(yH) = \text{Card}(xH) = \vert H\vert$.

Portanto, realizando os passos acima sucessivamente, criamos partições de $G$.

Como $G$ é finito, o processo terá finalizado após $n$ etapas.

Portanto, temos: $\vert G\vert = n\vert H\vert$. \qed

\textit{Observações:} 

\begin{enumerate}
	\item \textit{Segue como consequência direta do \textbf{Teorema de Lagrange} que caso $G$ seja um grupo finito e $\alpha \in G$, então $\mathcal{O}(\alpha)$ divide $\vert G\vert$.}
	\item \textit{Temos diretamente pela \hyperlink{def142}{\textbf{Definição 1.4.2}} que: $\vert G\vert = \vert H\vert [G:H]$.}
\end{enumerate}

\begin{proposition}
	Seja $G$ um grupo não finito e $H \leq G$. 

	Então vale o \hyperlink{lagrange}{\textbf{Teorema de Lagrange}}. 
\end{proposition}

\textbf{Demonstração:}

Demonstraremos novamente o \hyperlink{lagrange}{\textbf{Teorema de Lagrange}} de forma que a proposição acima possa ser justificada de forma clara. 

Seja $G$ um grupo e $H \leq G$.

Ora, sabemos que:

\[
\text{Ou} \; \; xH = yH \; \; \text{ou} \; \; xH \cap yH = \emptyset, \; \forall (x,y) \in G \times G.
\]

Sabemos também que, sendo $I$ um conjunto não vazio de índices tal que Card($I$) $= [G:H]$:

\[
\bigcup_{i \in I}^{\bullet} x_i H = G.
\]

Como $\vert G\vert$ é não finito, então se $\vert H\vert$ ou $[G:H]$ são não finitos, vale que $\vert G\vert = \vert H\vert[G:H]$.

Suponhamos, agora, que $\vert H\vert < \infty$ e $[G:H] < \infty$.

Como $[G:H] < \infty$, então $\vert I\vert < \infty$.

Logo, podemos escrever $I$ como:

\[
I = \{i_1, i_2, \dots, i_n, \; n\in \mathbb{N} \}.
\]

Logo:

\[
G = \bigcup_{i_1 \leq i \leq n}^{\bullet} x_i H.
\]

Portanto, podemos escrever:

\[
\vert G\vert = \sum_{k = 1}^{n} \vert x_{i_k} H\vert.
\]

Ora, deduzimos na demonstração do \hyperlink{langrange}{\textbf{Teorema de Lagrange}} que $\vert xH\vert = \vert H\vert, \; \forall x \in G$.

Portanto temos que:

\[
\vert G\vert = n\vert H\vert.
\]

Ora, mas $\vert G\vert$ é não finito e $\vert H\vert < \infty$. Absurdo !

Portanto, deduzimos que $\vert H\vert$ ou $[G:H]$ são não finitos.

Assim, provamos que o \hyperlink{lagrange}{\textbf{Teorema de Lagrange}} vale também para $\vert G\vert$ não finito. Isto é:

\[
\vert G\vert = \vert H\vert [G:H].
\]

\qed

\begin{proposition}
    Seja $G$ um grupo finito de ordem $p\in \mathbb{N}^*$. \\
    Se $p$ for primo, então $G$ é um grupo cíclico.
\end{proposition}

\textbf{Demonstração:}

Pelo \hyperlink{lagrange}{Teorema de Lagrange} sabemos que se $H$ é subgrupo de um grupo finito $G$, então $\vert H\vert$ divide $\vert G\vert$.

Como $\vert G\vert = p$ primo, então os únicos subgrupos possíveis de $G$ são seus subgrupos triviais.

Seja $x\in G$ tal que $x \neq e$, onde $e$ é o elemento neutro de $G$.

Logo, o único subgrupo gerado por $x$ é o próprio $G$, $\langle x\rangle = G$ \qed

\textit{Observação: como visto na \hyperlink{prop132}{\textbf{Proposição 1.3.2}}, $G$ também é abeliano!}

\hypertarget{teoremadeeuler}{}

\begin{theorem} 
	\textbf{Teorema de Euler (Grupos)} \\
	Seja $(G, \cdot)$ um grupo finito tal que $\vert G \vert = n, \; n\in \mathbb{Z}$.
	Então:
	\[
	\forall g \in G, \; g^n = 1.
	\]
\end{theorem}

\textbf{Demonstração:}

Seja $g$ um elemento do grupo finito $G$. Sabemos que $\langle g \rangle \leq G$. Sabemos também, pelo \hyperlink{lagrange}{\textbf{Teorema de Lagrange}} que $\mathcal{O}(g)$ divide a ordem de $G$. 

Ora, podemos então escrever:

\[
\vert G\vert = k\mathcal{O}(g), \; k\in \mathbb{Z} 
\]

Porém, pela \hyperlink{prop136}{\textbf{Proposição 1.3.6}}, deduzimos:

\[
g^{n} = g^{\vert G \vert} = g^{k\mathcal{O}(g)} = \left(g^{\mathcal{O}(g)}\right)^{k} = e^k = e
\]

Ora, demonstramos, com o argumento acima, sem perda de generalidade, tal fato para qualquer elemento de $G$.

\qed

\hypertarget{pequenofermat}{}

\begin{theorem} 
    \textbf{Pequeno Teorema de Fermat} \\
   	Seja $p$ um número primo e $a\in \mathbb{Z} \setminus p\mathbb{Z}$, então:
	\[
	a^{p-1} \equiv 1 \pmod{p}.
	\] 
\end{theorem}

\textbf{Demonstração:}

O \hyperlink{pequenofermat}{\textbf{Pequeno Teorema de Fermat}} é evidentemente o caso específico do \hyperlink{teoremadeeuler}{\textbf{Teorema de Euler}} em que $(G, \cdot) = ((\mathbb{Z}/p\mathbb{Z})^*, \odot)$.

\qed

\begin{proposition}
	Seja $G$ um grupo e sejam $K < H < G$.

	Logo $[G : K] = [G : H][H : K].$
\end{proposition}

\textbf{Demonstração:}

Basta aplicar sucessivamente o \hyperlink{lagrange}{\textbf{Teorema de Lagrange}}:

\begin{align*}
& H < G &&\Rightarrow && \lvert G \rvert = \lvert H \rvert \cdot [G : H] \; \; \; \text{(I)} \\
& K < H &&\Rightarrow && \lvert H \rvert = \lvert K \rvert \cdot [H : K] \; \; \; \text{(II)} \\
& K < G &&\Rightarrow && \lvert G \rvert = \lvert K \rvert \cdot [G : K] \; \; \; \text{(III)}
\end{align*}

Combinando as expressões (I) e (II), obtemos:

\[
\lvert G \rvert = \lvert K \rvert \cdot [H : K] \cdot [G : H] = \lvert K \rvert \cdot [G : K]
\]

Portanto:

\[
[G : K] = [G : H] \cdot [H : K]
\]

\qed

\section{Grupos Quocientes}

\begin{definition}
	Seja $G$ um grupo e $H \leq G$.

	Chamamos de \textbf{conjunto quociente} o conjunto $G/H$ (ou $\frac{G}{H}$) \textbf{cujos elementos são as classes laterais à esquerda (ou à direita) de $H$ em $G$}. 
\end{definition}

\textit{Observação: decorre da definição acima que $\vert G/H\vert = [G:H]$}.

\hypertarget{def152}{}

\begin{definition}
	Seja $G$ um grupo e $H\leq G$. 

	\textbf{Definimos a seguinte operação entre as classes laterais à esquerda de $H$ em $G$:}

\[
\bullet \; : 
\begin{array}{c}
G/H \times G/H \to G/H \\
(xH, yH)  \mapsto xyH
\end{array}
\]

\end{definition}

\textit{Observações:}

\begin{itemize}
	\item \textit{Note que, por convenção, estamos tratando das classes laterais à esquerda de $H$ em $G$. Entretanto, a definição é válida para classes laterais à direita também.}
	\item \textit{A operação definida é uma \textbf{lei de composição interna} por construção.}
	\item \textit{Não mostramos ainda que a operação está de fato bem definida, isto é:}
	
	\[
	\begin{cases}
	(x_1 H, y_1 H) \in G/H \times G/H \\	
	(x_2 H, y_2 H) \in G/H \times G/H
	\end{cases}
	\]
	\[
	\text{\textit{Se}} \; \; (x_1 H, y_1 H) = (x_2 H, y_2 H) \implies x_1 y_1 H = x_2 y_2 H.
	\]
	\textit{Tal fato será destacado na \hyperlink{prop151}{\textbf{Proposição 1.5.1}}.}
\end{itemize}

\hypertarget{prop151}{}

\begin{proposition}
Seja $G$ um grupo e $H\leq G$. 

As afirmações a seguir são equivalentes:

$(i)$ A operação definida na \hyperlink{def152}{\textbf{Definição 1.5.2}} está bem definida;

$(ii)$ $gHg^{-1} \subseteq H, \; \forall g \in G$;

$(iii)$ $gHg^{-1} = H, \; \forall g \in G$;

$(iv)$ $gH = Hg, \; \forall g \in G$.

\end{proposition}

\textbf{Demonstração:}

Mostremos as equivalências:

\[
(i) \iff (ii) \; \; \; \text{(I)}
\]

\[
(ii) \iff (iii) \; \; \; \text{(II)}
\]

\[
(iii) \iff (iv) \; \; \; \text{(III)}
\]

Comecemos mostrando a equivalência (I):

Ora, perceba que para $(x,y) \in G\times G$ e $(h, h')\in H\times H$, temos que:

$x$ e $xh$ são representantes distintos para a mesma classe lateral $xH$.

$y$ e $yh'$ são representantes distintos para a mesma classe lateral $yH$.

Portanto, podemos deduzir que a operação "$\bullet$" definida na \hyperlink{def152}{\textbf{Definição 1.5.2}} só estará bem definida se, e somente se:

\[
xyH = xhyh'H, \; \forall (x,y)\in G\times G \; \; \; \text{e} \; \; \; \forall (h,h')\in H\times H.
\]

Logo:

\[
xyH = xhyh'H \iff y^{-1}x^{-1}xyH = y^{-1}x^{-1}xhyh'H \iff H = y^{-1}hyh'H.
\]

Ora, mas isso é equivalente à dizer que a operação só estará bem definida se, e somente se:

\[
ghg^{-1} \in H,\; \forall g\in G, \; \forall h\in H.
\]

Com isso mostramos a equivalência (I).

Mostremos a equivalência (II):

Para a implicação $(ii) \implies (iii)$ mostremos que:

\[
gHg^{-1} \subseteq H \implies H \subseteq gHg^{-1}, \; \forall g \in G.
\]

Ora, temos diretamente da hipótese:

\[
gHg^{-1} \subseteq H \implies g^{-1}Hg \subseteq H \implies g(g^{-1}Hg)g^{-1} \subseteq gHg^{-1}
\]

\[
\implies H \subseteq gHg^{-1}
\]

Logo, podemos concluir que:

\[
gHg^{-1} \subseteq H \implies H = gHg^{-1}.
\]

A implicação $(iii) \implies (ii)$ é evidente.

Com isso mostramos a equivalência (II).

Mostremos a equivalência (III):

\[
gHg^{-1} = H \iff gHg^{-1}g = Hg \iff gH = Hg.
\]

Assim mostramos a equivalência (III).

Tendo mostrado as equivalências (I), (II) e (III), mostramos que todas as afirmações são duas a duas equivalentes. 

\qed

\begin{definition}
	Um subgrupo $H$ é um \textbf{subgrupo normal} de $G$ caso ele satisfaça as afirmações equivalentes da \hyperlink{prop151}{\textbf{Proposição 1.5.1}}.
	
	Neste caso, denotamos:

	\[
	H \trianglelefteq G
	\]

\end{definition}

\textit{Observações:}

\begin{itemize}
	\item \textit{Note que caso $H \trianglelefteq G$ então as classes laterais à esquerda e à direita de H são iguais;}
	\item \textit{Denotamos $H\triangleleft G$ se $H$ é um \textbf{subgrupo normal próprio de $G$}}.
	\item \textit{De forma geral quando queremos mostrar que um subgrupo $H$ é subgrupo normal de um grupo $G$, mostramos que $ghg^{-1} \in H$}.
\end{itemize}

\begin{example}
$G$ e $\{e\}$ (subgrupos triviais de $G$) são claramente subgrupos normais de $G$.
\end{example}

\begin{example}
Seja $G$ um grupo e $Z(G)$ o centro de $G$. Logo, $Z(G) \trianglelefteq G$.
\end{example}

\textbf{Demonstração:}

Já mostramos anteriormente que $Z(G) \leq G$.

Para mostrar que $Z(G) \trianglelefteq G$ basta mostrar que:

\[
\forall (g, z) \in G\times Z(G) \implies gzg^{-1} \in Z(G)
\]

Ora, mas pela própria definição de centro (todos elementos de G que comutam entre si), sabemos que:

\[
\text{Se } z\in Z(G) \implies zg = gz, \; \forall g \in G. 
\]

Logo:

\[
gzg^{-1} = zgg^{-1} = z \in Z(G). 
\]

\qed

\textit{Observaçãos:} 
\begin{itemize}
	\item \textit{De forma geral, é evidente que se $H \leq Z(G)$, então $H \trianglelefteq G$;}
	\item \textit{Isso equivale ainda a dizer que se $G$ é um grupo abeliano, então todos seus subgrupos são normais.}
\end{itemize}

\begin{definition}
\color{red}
Seja $G$ um grupo não-trivial. 

\textbf{Chamamos $G$ de grupo simples caso seus únicos subgrupos normais sejam $\{e\}$ e $G$}. 

Isto é, caso seus únicos subgrupos normais sejam os subgrupos triviais. 
\end{definition}

\begin{proposition}
Seja $G$ um grupo e $H\leq G$.

Se $[G:H] = 2$, então $H \trianglelefteq G$.
\end{proposition}

\textbf{Demonstração:}

Mostremos que $gH = Hg, \forall g \in G$. 

Demonstremos por disjunção de casos: 

Caso $g \in H$. Logo:

\[
gH = H = Hg
\]

Caso $g \notin H$. Logo:

Como $[G:H] = 2$, temos de imediato:

\[
G/H = \{H, gH\}
\]

Logo:

\[
G = H \dot{\cup} gH = H\dot{\cup} Hg
\]

Portanto, deduzimos de imediato que:

\[
gH = Hg
\]

Logo: $H \trianglelefteq G$.

\qed

\begin{definition}
\color{red}
Sejam $G$ um grupo e $A, B \leq G$. \textbf{Definimos o conjunto $AB$ da seguinte forma:}

\[
AB = \{ab, \; a\in A, \; b\in B \}.
\]

\end{definition}

\begin{proposition}
Seja $G$ um grupo e $H,K \leq G$. Logo:

\[
HK \; \; \text{é um subgrupo de $G$} \; \; \iff HK = KH.
\]
\end{proposition}

\textbf{Demonstração:}
{\color{red}
Mostremos a implicação ($\Longrightarrow$):

Seja $HK$ um subgrupo de $G$. 

Logo, temos que:

\[
(h, k) \in H\times K \implies hk\in HK \overset{HK\leq G}{\iff} (hk)^{-1}\in HK \iff k^{-1}h^{-1}\in HK.
\]

Ora, deduzimos, então que $KH = HK$.

Mostremos, agora, a implicação ($\Longleftarrow$):

Seja $HK = KH$, mostremos que $HK \leq G$.

Para mostrar que $HK \leq G$ é suficiente mostrar que:

\[
(HK)(HK) = HK
\]

\[
(HK)^{-1} = HK
\]

Note que essa é uma forma diferente, porém equivalente, de enunciar a \hyperlink{prop131}{\textbf{Proposição 1.3.1}}. 

Ora, temos diretamente que:

\[
(HK)(HK) = H(KH)K = H(HK)K = (HH)(KK) = HK
\]

\[
(HK)^{-1} = K^{-1}H^{-1} = KH = HK
\]

Com isso mostramos a proposição.

\qed}

\begin{corollary}
\color{red}
\hypertarget{col151}
Seja $G$ um grupo e $H, K \leq G$. Se $H \trianglelefteq G$ ou $K \trianglelefteq G$, então $HK \leq G$.
\end{corollary}

\textbf{Demonstração:}

{\color{red}

Sejam $H, K \leq G$. Tomemos $H$ como subgrupo normal de $G$ e mostremos, sem perda de generalidade, que $HK \leq G$.

Para mostrarmos que $HK \leq G$ é suficiente mostrar que a operação de $HK$ é uma lei de composição interna em $HK$ e que para todo elemento de $HK$, existe elemento inverso.

Note, primeiramente, que $HK$ é não vazio, uma vez que $H, K \leq G$.

Sejam $a, b \in HK$, mostremos que $ab \in HK$.

Ora, se $a, b \in HK$, então:

\[
a = hk, \; h\in H, \; k\in K
\]

\[
b = h'k', \; h'\in H, \; k'\in K
\]

Portanto, temos que:

\[
ab = hkh'k' = hkh'k^{-1}kk' = (h\underbrace{kh'k^{-1}}_{\in H})(kk') \in HK
\]

Mostremos, agora, que para $a\in HK, \; \exists a^{-1} \in HK$.

Ora, como $a \in HK$, então analogamente ao passo anterior temos que: 

\[
a = hk, \; h\in H, \; k\in K
\]

Portanto:

\[
a^{-1} = k^{-1}h^{-1} = k^{-1}h^{-1}kk^{-1} = (\underbrace{k^{-1}h^{-1}k}_{\in H})(k^{-1}) \in HK
\]

Portanto, mostramos que $HK \leq G$.

\qed}

\begin{corollary}
\color{red}
Seja $G$ um grupo e $H, K \trianglelefteq G$. Então $HK \trianglelefteq G$.
\end{corollary}

{\color{red}

\textbf{Demonstração:}

Sabemos a partir do \hyperlink{col151}{\textbf{Corolário 1.5.1}} que se $H \trianglelefteq G$ ou $K \trianglelefteq G$, temos que $HK \leq G$.

Mostremos que se $H\trianglelefteq G$ e $H\trianglelefteq G$, temos $HK \trianglelefteq G$.

Para isso, é suficiente mostrarmos que $gHKg^{-1} \in HK$.

Ora, como $H, K \trianglelefteq G$, temos:

\[
gHKg^{-1} = gHg^{-1}gKg^{-1} = (\underbrace{gHg^{-1}}_{\in H})(\underbrace{gKg^{-1}}_{\in K}) \in HK
\]

\qed}

\begin{proposition}
\color{red}
Seja $G$ um grupo finito e $H, K \leq G$. Então:

\[
\text{Card}(HK) = \frac{\vert H\vert \vert K\vert}{\vert H\cap K\vert}
\]
\end{proposition}

{\color{red}
\textbf{Demonstração:}



\qed}

\begin{definition}
Seja $G$ um grupo e $H\trianglelefteq G$.

\textbf{Então $(G/H, \bullet)$ é um grupo chamado de grupo quociente.}
\end{definition}

\textbf{Demonstração:}

Mostremos que $(G/H, \bullet)$ é de fato um grupo.

Ora, pela \hyperlink{def152}{\textbf{Definição 1.5.2}} sabemos que a operação "$\bullet$" é uma lei de composição interna por construção. Sabemos também, pela \hyperlink{prop151}{\textbf{Proposição 1.5.1}} que caso $H\trianglelefteq G$, então "$\bullet$" está bem definida. 

Nos resta mostrar que $(G/H, \bullet)$ satisfaz os axiomas de grupo.

De fato, $G/H$ é associativo em relação à operação "$\bullet$" pois:

Sejam $xH, yH, zH \in G/H$, temos:

\[
(xH\bullet yH)\bullet zH = (xy)H \bullet zH = (xy)zH \overset{(*)}{=} x(yz)H = xH\bullet(yH\bullet zH).
\]

Também temos evidentemente que o elemento neutro é dado por $H \in G/H$.

Ora, seja $xH\in G/H$, temos evidentemente:

\[
xH\bullet H = xH = H\bullet xH.
\]

Por fim, temos que para dado $xH \in G/H$, seu elemento inverso é dado por $x^{-1}H \in G/H$. 

De fato:

\[
xH\bullet x^{-1}H = xx^{-1}H = H = x^{-1}xH = x^{-1}H\bullet xH.
\]

Portanto, $(G/H, \bullet)$ é de fato um grupo.

\qed

\begin{example}
O grupo $\mathbb{Z}/n\mathbb{Z}$ é um grupo quociente formado pelo quociente entre $\mathbb{Z}$ e $n\mathbb{Z}\trianglelefteq \mathbb{Z}$.
\end{example}

\begin{proposition}
	Seja $G$ um grupo e $Z(G)$ seu centro. 

	Se o grupo $G/Z(G)$ for cíclico, então $G = Z(G)$.
\end{proposition}
	

\textbf{Demonstração:}

Seja $G/Z(G)$ um grupo cíclico. 

Mostremos que $G = Z(G)$, isto é, que $G$ é um grupo abeliano.

Como $G/Z(G)$ é cíclico, então existe $x\in G/Z(G)$ tal que $\langle x\rangle = G$.

Portanto, sabemos que, para determinado inteiro $k\in \mathbb{Z}$, temos:

\[
gz' = x^k, \; (g,z')\in G\times Z(G)
\]

Podemos reescrever a expressão acima como:

\[
g = x^kz, \; (g,z)\in G\times Z(G)
\]

Tomemos $g_1, g_2 \in G$ tal que:

\[
g_1 \overset{def}{=} x^{k_1}z_1, \; (k_1, z_1) \in \mathbb{Z}\times Z(G)
\]

\[
g_2 \overset{def}{=} x^{k_2}z_2, \; (k_2, z_2) \in \mathbb{Z}\times Z(G)
\]

Portanto temos:

\[
g_1g_2 = x^{k_1}z_1x^{k_2}z_2 = x^{k_1 + k_2}z_1z_2 = x^{k_2}x^{k_1}z_1z_2 = x^{k_2}z_2x^{k_1}z_1 = g_2g_1. 
\]

Isso se dá pelo fato de qualquer elemento de $z\in Z(G)$ comutar com elementos de $G$ e, portanto, elementos de $G/Z(G)$.

Com isso, mostramos que $G$ é um grupo abeliano e, portanto, $G = Z(G)$. 

\qed

\section{Homomorfismos de Grupos}

\begin{definition}
\color{red}
Sejam $(G, \cdot)$ e $(\mathcal{G}, *)$ dois grupos. \textbf{Uma função $\phi : G \rightarrow \mathcal{G}$ é chamada de homomorfismo de grupos se:}

\[
\phi(a\cdot b) = \phi(a)*\phi(b), \; \forall a,b \in G
\]

\end{definition}


\end{document}
